%Please use LuaLaTeX or XeLaTeX
\documentclass[11pt,aspectratio=169]{beamer}

\title[Published papers]{Published papers}
\date[2025/11/04]{Subdense final meeting}
\author[JR]{Raimbault et al.}
\institute[IGN-ENSG]{LaSTIG, Univ. Gustave Eiffel, IGN-ENSG}

\usetheme{lastig}

\colorlet{titlefgcolor}{LASTIGBlue}
\colorlet{accentcolor}{LASTIGRed}

\begin{document}

%\titleframe


\begin{frame}[fragile]{Benchmark of matching algorithms}

\footnotesize


\textbf{Published as } Guardiola, P., Raimbault, J., Olteanu-Raimond, A. M., and Perret, J. (2024). Benchmarking algorithms for matching geospatial vector data. In \textit{Proceedings of the French Regional Conference on Complex Systems} (pp. 277-280).


\begin{center}
	
	\includegraphics[width=0.35\linewidth]{figures/result_algo.png}\hspace{0.5cm}
	\includegraphics[width=0.4\linewidth,height=0.35\textheight]{figures/benchmark.png}	
	
\end{center}

Bi-objective benchmark optimising for \textbf{performance} (F-score on a $\simeq$2k buildings ground truth dataset) and \textbf{runtime}, for different parametrisations of two algorithms (multi-criteria and Geometric Matching of Areas): (i) various algorithm performances; (ii) under-detection of change by GMoA and over-detection by MCA; (iii) GMA better for m-n links, MCA better for 1-1 links.

$\rightarrow$ need fine-tuning to each case, region and urban morphology; or a \textbf{new approach combining both}.

\end{frame}


\end{document}